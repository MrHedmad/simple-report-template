\section{Introduction}

\gls{foobar} is a strange animal. \gls{foo} is another strange animal. It is true, since Albert said it \cite{AlbertRelativity}. Is any case, the \gls{gcd} is really useful if you need to calculate the \gls{gcd} of something.

\texttt{This text is monospace.} This text is boring. \textit{This text is in italics instead!}

% R code block is the default
% Override with `style=...`, like `style=PyStyle` for Python,
% or `style=ShellStyle` for shell
\begin{lstlisting}[caption=This is a listing]
get_db_names <- function(db_namespace) {
  suppressWarnings({
    if (!require(db_namespace, character.only = TRUE)) {
      BiocManager::install(db_namespace)
      suppressPackageStartupMessages(library(db_namespace, character.only = TRUE))
    }
  })
  possibilities <- ls(paste0("package:", db_namespace))
  db_name = gsub("\\.db", "", db_namespace)
  possibilities <- sapply(
    possibilities, gsub,
    pattern = db_name, replacement = ""
  )
  
  # Clean out the things that start with _ or . as they are functions and junk
  possibilities <- sapply(
    possibilities, gsub,
    pattern = "^[_\\.].*", replacement = "")
  
  possibilities <- possibilities[possibilities != ""]
  names(possibilities) <- NULL
  return(possibilities)
}
\end{lstlisting}

\begin{figure}[H]
    \centering
    \includegraphics[width=0.3\textwidth]{resources/images/sylveon.jpg}
    \caption{This is a magical pokemon, and one of my favourites!}
    \label{fig:pokemon}
\end{figure}

Figure \ref{fig:pokemon} shows my favourite pokemon.